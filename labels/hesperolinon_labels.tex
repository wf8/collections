
\documentclass[letterpaper,10pt]{article}

\usepackage[document]{ragged2e}
\usepackage{multicol}
\usepackage{textcomp}
\usepackage[cm]{fullpage}

\begin{document}
\pagenumbering{gobble}

%
% row begin
%

% label start
\begin{minipage}[t]{0.40\textwidth}

\begin{center}
University of California Herbarium \\
\begin{large}
Plants of Marin County, California \\
\end{large}
\vspace{\baselineskip}
\textbf{Linaceae} \\
\textit{Hesperolinon congestum} A. Gray\\
\end{center}

\begin{footnotesize}

\begin{multicols}{2}
37.882\textdegree N 122.455\textdegree W
\columnbreak
\begin{flushright}
Elev. 64 m
\end{flushright}
\end{multicols}

Old St. Hillary's preserve, accessed using Vistazo Street off of Lyford Drive from Tiburon Blvd.
\vspace{\baselineskip}

Flowering on southwest hillside in serpentine grassland. Y.P. Springer (2007) population \#0601.

\begin{multicols}{2}
W.A. Freyman 214 \\
with A.C. Schneider \& Y.P. Springer
\columnbreak
\begin{flushright}
17 May 2014
\end{flushright}
\end{multicols}

\end{footnotesize}

\end{minipage}
% label end
%
\hspace{2cm}
%
% label start
\begin{minipage}[t]{0.40\textwidth}

\begin{center}
University of California Herbarium \\
\begin{large}
Plants of Marin County, California \\
\end{large}
\vspace{\baselineskip}
\textbf{Linaceae} \\
\textit{Hesperolinon congestum} A. Gray\\
\end{center}

\begin{footnotesize}

\begin{multicols}{2}
37.912\textdegree N 122.487\textdegree W
\columnbreak
\begin{flushright}
Elev. 152 m
\end{flushright}
\end{multicols}

Ring Mountain preserve, accessed at end of Taylor Roadd off of Paradise Drive from Tiburon Blvd.
\vspace{\baselineskip}

In flower on serpentine grassland with \textit{Calochortus tiburonsis, Allium, Triteleia laxa, Eriogonum}. Y.P. Springer (2007) population \#0602.

\begin{multicols}{2}
W.A. Freyman 215 \\
with A.C. Schneider \& Y.P. Springer
\columnbreak
\begin{flushright}
17 May 2014
\end{flushright}
\end{multicols}

\end{footnotesize}

\end{minipage}
% label end

\vspace{2cm}
%
% row end
%

%
% row begin
%

% label start
\begin{minipage}[t]{0.40\textwidth}

\begin{center}
University of California Herbarium \\
\begin{large}
Plants of Marin County, California \\
\end{large}
\vspace{\baselineskip}
\textbf{Linaceae} \\
\textit{Hesperolinon micranthum} A. Gray\\
\end{center}

\begin{footnotesize}

\begin{multicols}{2}
37.969\textdegree N 122.629\textdegree W
\columnbreak
\begin{flushright}
Elev. 357 m
\end{flushright}
\end{multicols}

Near top of Fairfax/Bolinas Road off of Sir Francis Drake Blvd, along northern extension of Azalea Hill Trail (Pine Mountain Road).
\vspace{\baselineskip}

In flower on serpentine ridge with \textit{Arctostaphylos}. Y.P. Springer (2007) population \#1001.

\begin{multicols}{2}
W.A. Freyman 216 \\
with A.C. Schneider \& Y.P. Springer
\columnbreak
\begin{flushright}
17 May 2014
\end{flushright}
\end{multicols}

\end{footnotesize}

\end{minipage}
% label end
%
\hspace{2cm}
%
% label start
\begin{minipage}[t]{0.40\textwidth}

\begin{center}
University of California Herbarium \\
\begin{large}
Plants of Marin County, California \\
\end{large}
\vspace{\baselineskip}
\textbf{Linaceae} \\
\textit{Hesperolinon congestum} A. Gray\\
\end{center}

\begin{footnotesize}

\begin{multicols}{2}
37.969\textdegree N 122.629\textdegree W
\columnbreak
\begin{flushright}
Elev. 357 m
\end{flushright}
\end{multicols}

Near top of Fairfax/Bolinas Road off of Sir Francis Drake Blvd, along northern extension of Azalea Hill Trail (Pine Mountain Road).
\vspace{\baselineskip}

In flower on serpentine ridge with \textit{Arctostaphylos}. Y.P. Springer (2007) population \#0603.

\begin{multicols}{2}
W.A. Freyman 217 \\
with A.C. Schneider \& Y.P. Springer
\columnbreak
\begin{flushright}
17 May 2014
\end{flushright}
\end{multicols}

\end{footnotesize}

\end{minipage}
% label end

\vspace{2cm}
%
% row end
%

%
% row begin
%

% label start
\begin{minipage}[t]{0.40\textwidth}

\begin{center}
University of California Herbarium \\
\begin{large}
Plants of Marin County, California \\
\end{large}
\vspace{\baselineskip}
\textbf{Linaceae} \\
\textit{Hesperolinon congestum} A. Gray\\
\end{center}

\begin{footnotesize}

\begin{multicols}{2}
38.047\textdegree N 122.622\textdegree W
\columnbreak
\begin{flushright}
Elev. 210 m
\end{flushright}
\end{multicols}

At Big Rock along Lucas Valley Road off of Hwy 101 - along Loma Alta Fire Road.
\vspace{\baselineskip}

In flower on steep grassy slope with \textit{Triteleia laxa and Achillea}. Y.P. Springer (2007) population \#0604.

\begin{multicols}{2}
W.A. Freyman 218 \\
with A.C. Schneider \& Y.P. Springer
\columnbreak
\begin{flushright}
17 May 2014
\end{flushright}
\end{multicols}

\end{footnotesize}

\end{minipage}
% label end
%
\hspace{2cm}
%
% label start
\begin{minipage}[t]{0.40\textwidth}

\begin{center}
University of California Herbarium \\
\begin{large}
Plants of Napa County, California \\
\end{large}
\vspace{\baselineskip}
\textbf{Linaceae} \\
\textit{Hesperolinon californicum} Benth.\\
\end{center}

\begin{footnotesize}

\begin{multicols}{2}
38.557\textdegree N 122.371\textdegree W
\columnbreak
\begin{flushright}
Elev. 306 m
\end{flushright}
\end{multicols}

Along Pope Valley Road.
\vspace{\baselineskip}

Not yet flowering, in grassland with \textit{Sisyrinchium, Clarkia, Triteleia laxa, Delphinium}. Y.P. Springer (2007) population \#0413.

\begin{multicols}{2}
W.A. Freyman 219 \\
with A.C. Schneider \& Y.P. Springer
\columnbreak
\begin{flushright}
17 May 2014
\end{flushright}
\end{multicols}

\end{footnotesize}

\end{minipage}
% label end

\vspace{2cm}
%
% row end
%

%
% row begin
%

% label start
\begin{minipage}[t]{0.40\textwidth}

\begin{center}
University of California Herbarium \\
\begin{large}
Plants of Napa County, California \\
\end{large}
\vspace{\baselineskip}
\textbf{Linaceae} \\
\textit{Hesperolinon spergulinum} A. Gray\\
\end{center}

\begin{footnotesize}

\begin{multicols}{2}
38.566\textdegree N 122.398\textdegree W
\columnbreak
\begin{flushright}
Elev. 287 m
\end{flushright}
\end{multicols}

Along private road that runs west off of Pope Valley Road.
\vspace{\baselineskip}

Not yet flowering, on patchy barren soil with \textit{Arctostaphylos, Clarkia, Sisyrinchium}. Y.P. Springer (2007) population \#1204.

\begin{multicols}{2}
W.A. Freyman 221 \\
with A.C. Schneider \& Y.P. Springer
\columnbreak
\begin{flushright}
17 May 2014
\end{flushright}
\end{multicols}

\end{footnotesize}

\end{minipage}
% label end
%
\hspace{2cm}
%
% label start
\begin{minipage}[t]{0.40\textwidth}

\begin{center}
University of California Herbarium \\
\begin{large}
Plants of Napa County, California \\
\end{large}
\vspace{\baselineskip}
\textbf{Linaceae} \\
\textit{Hesperolinon sharsmithiae} R. O'Donnell\\
\end{center}

\begin{footnotesize}

\begin{multicols}{2}
38.654\textdegree N 122.359\textdegree W
\columnbreak
\begin{flushright}
Elev. 331 m
\end{flushright}
\end{multicols}

Along trail up to meadow above the Shadhiliyya Sufi Center.
\vspace{\baselineskip}

In flower. Y.P. Springer (2007) population \#1102.

\begin{multicols}{2}
W.A. Freyman 224 \\
with A.C. Schneider \& Y.P. Springer
\columnbreak
\begin{flushright}
17 May 2014
\end{flushright}
\end{multicols}

\end{footnotesize}

\end{minipage}
% label end

\vspace{2cm}
%
% row end
%

%
% row begin
%

% label start
\begin{minipage}[t]{0.40\textwidth}

\begin{center}
University of California Herbarium \\
\begin{large}
Plants of Napa County, California \\
\end{large}
\vspace{\baselineskip}
\textbf{Linaceae} \\
\textit{Hesperolinon californicum} Benth.\\
\end{center}

\begin{footnotesize}

\begin{multicols}{2}
38.657\textdegree N 122.360\textdegree W
\columnbreak
\begin{flushright}
Elev. 380 m
\end{flushright}
\end{multicols}

In meadow above Shadhiliyya Sufi Center at Walter Springs (accessed via turnoff from Pope Canyon Road).
\vspace{\baselineskip}

Not yet flowering, on serpentine grassland with \textit{Sisyrinchium, Calochortus, Dichelostemma, Delphinium}. Y.P. Springer (2007) population \#0410.

\begin{multicols}{2}
W.A. Freyman 225 \\
with A.C. Schneider \& Y.P. Springer
\columnbreak
\begin{flushright}
17 May 2014
\end{flushright}
\end{multicols}

\end{footnotesize}

\end{minipage}
% label end
%
\hspace{2cm}
%
% label start
\begin{minipage}[t]{0.40\textwidth}

\begin{center}
University of California Herbarium \\
\begin{large}
Plants of Lake County, California \\
\end{large}
\vspace{\baselineskip}
\textbf{Linaceae} \\
\textit{Hesperolinon clevelandii} Greene\\
\end{center}

\begin{footnotesize}

\begin{multicols}{2}
38.828\textdegree N 122.534\textdegree W
\columnbreak
\begin{flushright}
Elev. 546 m
\end{flushright}
\end{multicols}

Along private dirt driveway, north side of Jerusalem Grade Road east of junction with Spruce Grove Road.
\vspace{\baselineskip}

In flower with \textit{Salvia sonomensis} (common associate) and \textit{Ceanothus}. Y.P. Springer (2007) population \#0503.

\begin{multicols}{2}
W.A. Freyman 227 \\
with A.C. Schneider \& Y.P. Springer
\columnbreak
\begin{flushright}
17 May 2014
\end{flushright}
\end{multicols}

\end{footnotesize}

\end{minipage}
% label end

\vspace{2cm}
%
% row end
%

%
% row begin
%

% label start
\begin{minipage}[t]{0.40\textwidth}

\begin{center}
University of California Herbarium \\
\begin{large}
Plants of Napa County, California \\
\end{large}
\vspace{\baselineskip}
\textbf{Linaceae} \\
\textit{Hesperolinon californicum} Benth.\\
\end{center}

\begin{footnotesize}

\begin{multicols}{2}
38.610\textdegree N 122.347\textdegree W
\columnbreak
\begin{flushright}
Elev. 195 m
\end{flushright}
\end{multicols}

Along dirt road offshoot from Dollarhide Road, which originates on Hardin Road just north of Wantrup Wildlife refuge - access is through St. Suprey vineyards.
\vspace{\baselineskip}

In flower on grassy serpentine hillside in pasture. Y.P. Springer (2007) population \#0412.

\begin{multicols}{2}
W.A. Freyman 228 \\
with A.C. Schneider \& Y.P. Springer
\columnbreak
\begin{flushright}
18 May 2014
\end{flushright}
\end{multicols}

\end{footnotesize}

\end{minipage}
% label end
%
\hspace{2cm}
%
% label start
\begin{minipage}[t]{0.40\textwidth}

\begin{center}
University of California Herbarium \\
\begin{large}
Plants of Napa County, California \\
\end{large}
\vspace{\baselineskip}
\textbf{Linaceae} \\
\textit{Hesperolinon californicum} Benth.\\
\end{center}

\begin{footnotesize}

\begin{multicols}{2}
38.630\textdegree N 122.412\textdegree W
\columnbreak
\begin{flushright}
Elev. 208 m
\end{flushright}
\end{multicols}

At end of Barnett Road off of Pope Valley Road.
\vspace{\baselineskip}

Flowering in grassy pasture. Y.P. Springer (2007) population \#0411.

\begin{multicols}{2}
W.A. Freyman 229 \\
with A.C. Schneider \& Y.P. Springer
\columnbreak
\begin{flushright}
18 May 2014
\end{flushright}
\end{multicols}

\end{footnotesize}

\end{minipage}
% label end

\vspace{2cm}
%
% row end
%

%
% row begin
%

% label start
\begin{minipage}[t]{0.40\textwidth}

\begin{center}
University of California Herbarium \\
\begin{large}
Plants of Lake County, California \\
\end{large}
\vspace{\baselineskip}
\textbf{Linaceae} \\
\textit{Hesperolinon bicarpellatum} H. Sharsm.\\
\end{center}

\begin{footnotesize}

\begin{multicols}{2}
38.756\textdegree N 122.570\textdegree W
\columnbreak
\begin{flushright}
Elev. 365 m
\end{flushright}
\end{multicols}

Across field to the north near the end of Loconomi Road off of Butts Canyon Road.
\vspace{\baselineskip}

In flower on serpentine slope near \textit{Hesperolinon californicum} with \textit{Eriodictyon} and \textit{Orobanche fasciculata}. Y.P. Springer (2007) population \#0203.

\begin{multicols}{2}
W.A. Freyman 230 \\
with A.C. Schneider \& Y.P. Springer
\columnbreak
\begin{flushright}
18 May 2014
\end{flushright}
\end{multicols}

\end{footnotesize}

\end{minipage}
% label end
%
\hspace{2cm}
%
% label start
\begin{minipage}[t]{0.40\textwidth}

\begin{center}
University of California Herbarium \\
\begin{large}
Plants of Lake County, California \\
\end{large}
\vspace{\baselineskip}
\textbf{Linaceae} \\
\textit{Hesperolinon californicum} Benth.\\
\end{center}

\begin{footnotesize}

\begin{multicols}{2}
38.763\textdegree N 122.563\textdegree W
\columnbreak
\begin{flushright}
Elev. 315 m
\end{flushright}
\end{multicols}

On 3B ranch, accessed via Grange Road off of Hwy 29 east of Middletown.
\vspace{\baselineskip}

In flower on grassy serpentine slope. Y.P. Springer (2007) population \#0408.

\begin{multicols}{2}
W.A. Freyman 231 \\
with A.C. Schneider \& Y.P. Springer
\columnbreak
\begin{flushright}
18 May 2014
\end{flushright}
\end{multicols}

\end{footnotesize}

\end{minipage}
% label end

\vspace{2cm}
%
% row end
%

%
% row begin
%

% label start
\begin{minipage}[t]{0.40\textwidth}

\begin{center}
University of California Herbarium \\
\begin{large}
Plants of Lake County, California \\
\end{large}
\vspace{\baselineskip}
\textbf{Linaceae} \\
\textit{Hesperolinon spergulinum} A. Gray\\
\end{center}

\begin{footnotesize}

\begin{multicols}{2}
38.731\textdegree N 122.661\textdegree W
\columnbreak
\begin{flushright}
Elev. 435 m
\end{flushright}
\end{multicols}

Off of Dr Creek Road southwest of Middletown, accessed from Dry Creek Cuttoff off of Hwy 29 heading south out of Middletown.
\vspace{\baselineskip}

Not yet flowering, on slope with \textit{Arctostaphylos} and \textit{Toxicodendron}. Y.P. Springer (2007) population \#1201.

\begin{multicols}{2}
W.A. Freyman 234 \\
with A.C. Schneider \& Y.P. Springer
\columnbreak
\begin{flushright}
18 May 2014
\end{flushright}
\end{multicols}

\end{footnotesize}

\end{minipage}
% label end
%
\hspace{2cm}
%
% label start
\begin{minipage}[t]{0.40\textwidth}

\begin{center}
University of California Herbarium \\
\begin{large}
Plants of Lake County, California \\
\end{large}
\vspace{\baselineskip}
\textbf{Linaceae} \\
\textit{Hesperolinon didymocarpum} H. Sharsm.\\
\end{center}

\begin{footnotesize}

\begin{multicols}{2}
38.790\textdegree N 122.612\textdegree W
\columnbreak
\begin{flushright}
Elev. 343 m
\end{flushright}
\end{multicols}

On small hill on east side of Big Canyon Road along side of road, owned by Diamond D Ranch.
\vspace{\baselineskip}

Not yet flowering, on serpentine slope. Y.P. Springer (2007) population \#0702.

\begin{multicols}{2}
W.A. Freyman 235 \\
with A.C. Schneider \& Y.P. Springer
\columnbreak
\begin{flushright}
18 May 2014
\end{flushright}
\end{multicols}

\end{footnotesize}

\end{minipage}
% label end

\vspace{2cm}
%
% row end
%

%
% row begin
%

% label start
\begin{minipage}[t]{0.40\textwidth}

\begin{center}
University of California Herbarium \\
\begin{large}
Plants of Lake County, California \\
\end{large}
\vspace{\baselineskip}
\textbf{Linaceae} \\
\textit{Hesperolinon spergulinum} A. Gray\\
\end{center}

\begin{footnotesize}

\begin{multicols}{2}
38.778\textdegree N 122.675\textdegree W
\columnbreak
\begin{flushright}
Elev. 426 m
\end{flushright}
\end{multicols}

19622 Blossom Springs Ranch off of Hwy 175 just east of junction with Anderson Springs Road.
\vspace{\baselineskip}

Flowering on serpentine gravel slope. Y.P. Springer (2007) population \#1202.

\begin{multicols}{2}
W.A. Freyman 273 \\
with A.C. Schneider \& Y.P. Springer
\columnbreak
\begin{flushright}
18 May 2014
\end{flushright}
\end{multicols}

\end{footnotesize}

\end{minipage}
% label end
%
\hspace{2cm}
%
% label start
\begin{minipage}[t]{0.40\textwidth}

\begin{center}
University of California Herbarium \\
\begin{large}
Plants of Lake County, California \\
\end{large}
\vspace{\baselineskip}
\textbf{Linaceae} \\
\textit{Hesperolinon adenophyllum} A. Gray\\
\end{center}

\begin{footnotesize}

\begin{multicols}{2}
38.778\textdegree N 122.675\textdegree W
\columnbreak
\begin{flushright}
Elev. 426 m
\end{flushright}
\end{multicols}

East side of Bottle Rock Road just north of junction with Spring Hill Road.
\vspace{\baselineskip}

On serpentine gravel slope, not yet flowering. Y.P. Springer (2007) population \#0106.

\begin{multicols}{2}
W.A. Freyman 274 \\
with A.C. Schneider \& Y.P. Springer
\columnbreak
\begin{flushright}
18 May 2014
\end{flushright}
\end{multicols}

\end{footnotesize}

\end{minipage}
% label end

\vspace{2cm}
%
% row end
%

%
% row begin
%

% label start
\begin{minipage}[t]{0.40\textwidth}

\begin{center}
University of California Herbarium \\
\begin{large}
Plants of Lake County, California \\
\end{large}
\vspace{\baselineskip}
\textbf{Linaceae} \\
\textit{Hesperolinon clevelandii} Greene\\
\end{center}

\begin{footnotesize}

\begin{multicols}{2}
38.920\textdegree N 122.704\textdegree W
\columnbreak
\begin{flushright}
Elev. 576 m
\end{flushright}
\end{multicols}

North side of Highway 29 east of Doten Road.
\vspace{\baselineskip}

In flower with \textit{Arctostaphylos, Eriodictyon, Adenostoma}, and \textit{Salvia sonomensis}. Y.P. Springer (2007) population \# 050.

\begin{multicols}{2}
W.A. Freyman 278 \\
with A.C. Schneider \& Y.P. Springer
\columnbreak
\begin{flushright}
18 May 2014
\end{flushright}
\end{multicols}

\end{footnotesize}

\end{minipage}
% label end
%
\hspace{2cm}
%
% label start
\begin{minipage}[t]{0.40\textwidth}

\begin{center}
University of California Herbarium \\
\begin{large}
Plants of Napa County, California \\
\end{large}
\vspace{\baselineskip}
\textbf{Linaceae} \\
\textit{Hesperolinon disjunctum} H. Sharsm.\\
\end{center}

\begin{footnotesize}

\begin{multicols}{2}
38.792\textdegree N 122.353\textdegree W
\columnbreak
\begin{flushright}
Elev. 491 m
\end{flushright}
\end{multicols}

Knoxville Devilhead Road.
\vspace{\baselineskip}

In full bloom on rocky serpentine slope under \textit{Arctostaphylos and Cupressus}, and with \textit{Hesperolinon drymarioides} (not in bloom).

\begin{multicols}{2}
W.A. Freyman 279 \\
with A.C. Schneider \& Y.P. Springer
\columnbreak
\begin{flushright}
19 May 2014
\end{flushright}
\end{multicols}

\end{footnotesize}

\end{minipage}
% label end

\vspace{2cm}
%
% row end
%

%
% row begin
%

% label start
\begin{minipage}[t]{0.40\textwidth}

\begin{center}
University of California Herbarium \\
\begin{large}
Plants of Lake County, California \\
\end{large}
\vspace{\baselineskip}
\textbf{Linaceae} \\
\textit{Hesperolinon californicum} Benth.\\
\end{center}

\begin{footnotesize}

\begin{multicols}{2}
38.863\textdegree N 122.400\textdegree W
\columnbreak
\begin{flushright}
Elev. 708 m
\end{flushright}
\end{multicols}

Along ridge at top of dirt access road that goes north off of Morgan Valley Road.
\vspace{\baselineskip}

Flowering and abundant on serpentine grass and gravel slope. Y.P. Springer (2007) population \#0404.

\begin{multicols}{2}
W.A. Freyman 281 \\
with A.C. Schneider \& Y.P. Springer
\columnbreak
\begin{flushright}
19 May 2014
\end{flushright}
\end{multicols}

\end{footnotesize}

\end{minipage}
% label end
%
\hspace{2cm}
%
% label start
\begin{minipage}[t]{0.40\textwidth}

\begin{center}
University of California Herbarium \\
\begin{large}
Plants of Lake County, California \\
\end{large}
\vspace{\baselineskip}
\textbf{Linaceae} \\
\textit{Hesperolinon californicum} Benth.\\
\end{center}

\begin{footnotesize}

\begin{multicols}{2}
38.872\textdegree N 122.405\textdegree W
\columnbreak
\begin{flushright}
Elev. 689 m
\end{flushright}
\end{multicols}

Along Reiff Road east of junction with Morgan Valley Road.
\vspace{\baselineskip}

In flower on grassy serpentine slope. Y.P. Springer (2007) population \#0403.

\begin{multicols}{2}
W.A. Freyman 283 \\
with A.C. Schneider \& Y.P. Springer
\columnbreak
\begin{flushright}
19 May 2014
\end{flushright}
\end{multicols}

\end{footnotesize}

\end{minipage}
% label end

\vspace{2cm}
%
% row end
%

%
% row begin
%

% label start
\begin{minipage}[t]{0.40\textwidth}

\begin{center}
University of California Herbarium \\
\begin{large}
Plants of Tehama County, California \\
\end{large}
\vspace{\baselineskip}
\textbf{Linaceae} \\
\textit{Hesperolinon tehamense} H. Sharsm.\\
\end{center}

\begin{footnotesize}

\begin{multicols}{2}
39.890\textdegree N 122.634\textdegree W
\columnbreak
\begin{flushright}
Elev. 709 m
\end{flushright}
\end{multicols}

Near beginning of incline on Toomes Camp Road west of Paskenta.
\vspace{\baselineskip}

In flower on serpentine slope with \textit{Castilleja} under \textit{Adenostoma}. Y.P. Springer (2007) population \#1303.

\begin{multicols}{2}
W.A. Freyman 284 \\
with A.C. Schneider \& Y.P. Springer
\columnbreak
\begin{flushright}
20 May 2014
\end{flushright}
\end{multicols}

\end{footnotesize}

\end{minipage}
% label end
%
\hspace{2cm}
%
% label start
\begin{minipage}[t]{0.40\textwidth}

\begin{center}
University of California Herbarium \\
\begin{large}
Plants of Tehama County, California \\
\end{large}
\vspace{\baselineskip}
\textbf{Linaceae} \\
\textit{Hesperolinon tehamense} H. Sharsm.\\
\end{center}

\begin{footnotesize}

\begin{multicols}{2}
39.894\textdegree N 122.647\textdegree W
\columnbreak
\begin{flushright}
Elev. 872 m
\end{flushright}
\end{multicols}

About half way up to radio tower on Toomes Camp Road west of Paskenta.
\vspace{\baselineskip}

In flower under \textit{Adenostoma} on serpentine slope. Y.P. Springer (2007) population \#1304.

\begin{multicols}{2}
W.A. Freyman 285 \\
with A.C. Schneider \& Y.P. Springer
\columnbreak
\begin{flushright}
20 May 2014
\end{flushright}
\end{multicols}

\end{footnotesize}

\end{minipage}
% label end

\vspace{2cm}
%
% row end
%

%
% row begin
%

% label start
\begin{minipage}[t]{0.40\textwidth}

\begin{center}
University of California Herbarium \\
\begin{large}
Plants of Tehama County, California \\
\end{large}
\vspace{\baselineskip}
\textbf{Linaceae} \\
\textit{Hesperolinon tehamense} H. Sharsm.\\
\end{center}

\begin{footnotesize}

\begin{multicols}{2}
39.829\textdegree N 122.637\textdegree W
\columnbreak
\begin{flushright}
Elev. 397 m
\end{flushright}
\end{multicols}

Along road to Mud Flat campground off of Round Valley Road southwest of Paskenta.
\vspace{\baselineskip}

In flower under \textit{Arctostaphylos} on serpentine slope. Y.P. Springer (2007) population \#1302.

\begin{multicols}{2}
W.A. Freyman 286 \\
with A.C. Schneider \& Y.P. Springer
\columnbreak
\begin{flushright}
20 May 2014
\end{flushright}
\end{multicols}

\end{footnotesize}

\end{minipage}
% label end
%
\hspace{2cm}
%
% label start
\begin{minipage}[t]{0.40\textwidth}

\begin{center}
University of California Herbarium \\
\begin{large}
Plants of Glenn County, California \\
\end{large}
\vspace{\baselineskip}
\textbf{Linaceae} \\
\textit{Hesperolinon disjunctum} H. Sharsm.\\
\end{center}

\begin{footnotesize}

\begin{multicols}{2}
39.413\textdegree N 122.588\textdegree W
\columnbreak
\begin{flushright}
Elev. 295 m
\end{flushright}
\end{multicols}

Along Forest Road 18N30 accessed via Black Diamond Road from Elk Creek Stonyford Road north of Lodoga.
\vspace{\baselineskip}

In flower with \textit{Adenostoma, Arctostaphylos, Quercus durata}. Y.P. Springer (2007) population \#0805.

\begin{multicols}{2}
W.A. Freyman 287 \\
with A.C. Schneider \& Y.P. Springer
\columnbreak
\begin{flushright}
20 May 2014
\end{flushright}
\end{multicols}

\end{footnotesize}

\end{minipage}
% label end

\vspace{2cm}
%
% row end
%

%
% row begin
%

% label start
\begin{minipage}[t]{0.40\textwidth}

\begin{center}
University of California Herbarium \\
\begin{large}
Plants of Colusa County, California \\
\end{large}
\vspace{\baselineskip}
\textbf{Linaceae} \\
\textit{Hesperolinon californicum} Benth.\\
\end{center}

\begin{footnotesize}

\begin{multicols}{2}
39.298\textdegree N 122.543\textdegree W
\columnbreak
\begin{flushright}
Elev. 416 m
\end{flushright}
\end{multicols}

Along north side of Goat Mountain Road west of Lodoga.
\vspace{\baselineskip}

Flowering in grassy field under oaks. Y.P. Springer (2007) population \#0401.

\begin{multicols}{2}
W.A. Freyman 288 \\
with A.C. Schneider \& Y.P. Springer
\columnbreak
\begin{flushright}
20 May 2014
\end{flushright}
\end{multicols}

\end{footnotesize}

\end{minipage}
% label end
%
\hspace{2cm}
%
% label start
\begin{minipage}[t]{0.40\textwidth}

\begin{center}
University of California Herbarium \\
\begin{large}
Plants of Colusa County, California \\
\end{large}
\vspace{\baselineskip}
\textbf{Linaceae} \\
\textit{Hesperolinon drymarioides} Curran\\
\end{center}

\begin{footnotesize}

\begin{multicols}{2}
39.257\textdegree N 122.529\textdegree W
\columnbreak
\begin{flushright}
Elev. 426 m
\end{flushright}
\end{multicols}

200 feet up slope from end of Cook Springs Road west off of Leesville Lodoga Road south of Lodoga.
\vspace{\baselineskip}

Not yet flowering, on steep rocky slope with \textit{Adenostoma} and \textit{Quercus durata}.

\begin{multicols}{2}
W.A. Freyman 289 \\
with A.C. Schneider \& Y.P. Springer
\columnbreak
\begin{flushright}
20 May 2014
\end{flushright}
\end{multicols}

\end{footnotesize}

\end{minipage}
% label end

\vspace{2cm}
%
% row end
%

%
% row begin
%

% label start
\begin{minipage}[t]{0.40\textwidth}

\begin{center}
University of California Herbarium \\
\begin{large}
Plants of Lake County, California \\
\end{large}
\vspace{\baselineskip}
\textbf{Linaceae} \\
\textit{Hesperolinon disjunctum} H. Sharsm.\\
\end{center}

\begin{footnotesize}

\begin{multicols}{2}
39.168\textdegree N 122.532\textdegree W
\columnbreak
\begin{flushright}
Elev. 476 m
\end{flushright}
\end{multicols}

Along north side of Bartlett Springs Road where it meets Indian Valley Reservoir by Stanton Creek.
\vspace{\baselineskip}

In flower on serpentine slope above reservoir with \textit{Arctostaphylos, Adenostoma, Quercus durata}. Y.P. Springer (2007) population \#0806.

\begin{multicols}{2}
W.A. Freyman 290 \\
with A.C. Schneider \& Y.P. Springer
\columnbreak
\begin{flushright}
20 May 2014
\end{flushright}
\end{multicols}

\end{footnotesize}

\end{minipage}
% label end
%
\hspace{2cm}
%
% label start
\begin{minipage}[t]{0.40\textwidth}

\begin{center}
University of California Herbarium \\
\begin{large}
Plants of Napa County, California \\
\end{large}
\vspace{\baselineskip}
\textbf{Linaceae} \\
\textit{Hesperolinon californicum} Benth.\\
\end{center}

\begin{footnotesize}

\begin{multicols}{2}
38.808\textdegree N 122.324\textdegree W
\columnbreak
\begin{flushright}
Elev. 409 m
\end{flushright}
\end{multicols}

East of Knoxville Berryessa Road at pullout by old chimney.
\vspace{\baselineskip}

In flower on serpentine slope above grassy pasture with \textit{Quercus durata}. Y.P. Springer (2007) population \#0407.

\begin{multicols}{2}
W.A. Freyman 291 \\
with A.C. Schneider \& Y.P. Springer
\columnbreak
\begin{flushright}
21 May 2014
\end{flushright}
\end{multicols}

\end{footnotesize}

\end{minipage}
% label end

\vspace{2cm}
%
% row end
%

%
% row begin
%

% label start
\begin{minipage}[t]{0.40\textwidth}

\begin{center}
University of California Herbarium \\
\begin{large}
Plants of Lake County, California \\
\end{large}
\vspace{\baselineskip}
\textbf{Linaceae} \\
\textit{Hesperolinon adenophyllum} A. Gray\\
\end{center}

\begin{footnotesize}

\begin{multicols}{2}
39.350\textdegree N 122.871\textdegree W
\columnbreak
\begin{flushright}
Elev. 611 m
\end{flushright}
\end{multicols}

Along north side of Forest Road 18N01 (M10) accessed from Elk Mountain Road (M1) north of Upper Lake, just north of Rice Creek.
\vspace{\baselineskip}

In flower on serpentine slope with \textit{Quercus durata}. Y.P. Springer (2007) population \#0104.

\begin{multicols}{2}
W.A. Freyman 292 \\
with A.C. Schneider \& Y.P. Springer
\columnbreak
\begin{flushright}
21 May 2014
\end{flushright}
\end{multicols}

\end{footnotesize}

\end{minipage}
% label end
%
\hspace{2cm}
%
% label start
\begin{minipage}[t]{0.40\textwidth}

\begin{center}
University of California Herbarium \\
\begin{large}
Plants of Lake County, California \\
\end{large}
\vspace{\baselineskip}
\textbf{Linaceae} \\
\textit{Hesperolinon adenophyllum} A. Gray\\
\end{center}

\begin{footnotesize}

\begin{multicols}{2}
39.390\textdegree N 122.948\textdegree W
\columnbreak
\begin{flushright}
Elev. 800 m
\end{flushright}
\end{multicols}

Along Packsaddle Trail access Forest Road 18N25, off of Elk Mountain Road (M1) out of Upper Lake.
\vspace{\baselineskip}

Not yet in flower, on serpentine slope with \textit{Arctostaphylos} and \textit{Adenostoma}. Y.P. Springer (2007) population \#0103.

\begin{multicols}{2}
W.A. Freyman 293 \\
with A.C. Schneider \& Y.P. Springer
\columnbreak
\begin{flushright}
21 May 2014
\end{flushright}
\end{multicols}

\end{footnotesize}

\end{minipage}
% label end

\vspace{2cm}
%
% row end
%

%
% row begin
%

% label start
\begin{minipage}[t]{0.40\textwidth}

\begin{center}
University of California Herbarium \\
\begin{large}
Plants of Lake County, California \\
\end{large}
\vspace{\baselineskip}
\textbf{Linaceae} \\
\textit{Hesperolinon adenophyllum} A. Gray\\
\end{center}

\begin{footnotesize}

\begin{multicols}{2}
39.410\textdegree N 122.956\textdegree W
\columnbreak
\begin{flushright}
Elev. 585 m
\end{flushright}
\end{multicols}

From Upper Lake, Elk Mountain Road (M1) out to Lake Pillsbury, then on the service road past the dam.
\vspace{\baselineskip}

Flowering on grassy serpentine slope above Lake Pillsbury. Y.P. Springer (2007) population \#0102.

\begin{multicols}{2}
W.A. Freyman 294 \\
with A.C. Schneider \& Y.P. Springer
\columnbreak
\begin{flushright}
21 May 2014
\end{flushright}
\end{multicols}

\end{footnotesize}

\end{minipage}
% label end
%
\hspace{2cm}
%
% label start
\begin{minipage}[t]{0.40\textwidth}

\begin{center}
University of California Herbarium \\
\begin{large}
Plants of Sonoma County, California \\
\end{large}
\vspace{\baselineskip}
\textbf{Linaceae} \\
\textit{Hesperolinon micranthum} A. Gray\\
\end{center}

\begin{footnotesize}

\begin{multicols}{2}
38.443\textdegree N 122.533\textdegree W
\columnbreak
\begin{flushright}
Elev. 416 m
\end{flushright}
\end{multicols}

Goodspeed Trail, Sugarloaf Ridge S.P.
\vspace{\baselineskip}

Flowering on serpentine slope with \textit{Adenostoma, Arctostphylos, Pickeringia}. Y.P. Springer (2007) population \#1005.

\begin{multicols}{2}
W.A. Freyman 298 \\
with A.C. Schneider \& Y.P. Springer
\columnbreak
\begin{flushright}
21 May 2014
\end{flushright}
\end{multicols}

\end{footnotesize}

\end{minipage}
% label end

\vspace{2cm}
%
% row end
%


\end{document}
